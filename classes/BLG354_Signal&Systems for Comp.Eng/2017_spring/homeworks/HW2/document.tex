\documentclass{article}
\usepackage[utf8]{inputenc}
\usepackage{graphicx} 
\usepackage{amsmath} 
\setlength{\parindent}{4em}
\setlength{\parskip}{1em}
\renewcommand{\baselinestretch}{1.5}

\title{BLG 354E Homework - 2}
\author{Yunus Güngör }
\date{April 2018}

\begin{document}
	
	\maketitle
	
	\section{Answers}
	
	This homework only includes answers to given questions
	
	1)$$F(t)=\int_{-\infty}^{\infty}f(t)e^{-j2\pi ft}dt$$
	\par
	Consider a discreate signal $X[t]$ which is multiplication of a continuous signal $f(t)$ and a pulse signal $S(t)=\sum_{-\infty}^{\infty}\delta (t-nT_s)$. $T_s=$time between samples.
	$$F(X(t))=\int_{-\infty}^{\infty}[X(t)\sum_{-\infty}^{\infty}\delta(t-nT_s)]e^{-j2 \pi ft}dt$$
	$$F(X(t))=\sum_{-\infty}^{\infty}\int_{-\infty}^{\infty}[X(t)\delta(t-nT_s)]e^{-j2 \pi ft}dt$$
	$$X(t)=\sum_{n=-\infty}^{\infty}x(nT_s)e^{-j2 \pi f n T_s}$$
	\par
	$2 \pi f =w$ $nT_s=n$ (if function is considered as an array)
	$$X(e^{jw})=\sum_{n=0}^{N-1}x[n]e^{-jwn}$$
	\par
		Sampling $w$ at $w_k=2\pi k / N$ , $k=0,1,2,...N-1$
	$$X[k]=\sum_{n=0}^{N-1}x[n]e^{-j2\pi kn/N}$$
	\par
	
	2)Code for this question can be found in attachment.
	
	3)Unit Impulse, is the signal that has value $1$ at $t=1$ and has value $0$ otherwise. It is defined as $\delta(n) = \begin{cases}
	0  & \text{if } n \neq 0 \\
	1  & \text{if } n=0
	\end{cases} \quad$
	\par
	Unit impulse response is the output from a signal processing system for the input of unit impulse
	\par

	4)Linear and Time Invariant systems are both time invariant and linear. A system can be represented as $y[n]=H(x[n])$ and $y(n)=h(n)x(n)$. $y(n)$ is output, $h(n)$ is system function and $x(n)$ is the input.
	\par
	If a system is time invariant: $x(n-n_0)h(n)=y(n-n_0)$.
	\par
	If a system is linear and $$x_1(n)h(n)=y_1(n)$$ $$x_2(n)h(n)=y_2(n)$$ is given, then $$(ax_1(n)+bx_2(n)h(n)=ay_1(n)+by_2(n)$$
	\par
	$$x[n]=\sum_{-\infty}^{\infty}x[k]\delta[n-k]$$
	$$y[n]=H(\sum_{-\infty}^{\infty}x[k]\delta[n-k])=\sum_{-\infty}^{\infty}H(x[k]\delta[n-k])$$
	$$y[n]=\sum_{-\infty}^{\infty}x[k]H(\delta[n-k])=\sum_{-\infty}^{\infty}x[k]h[n-k]$$
	
	a)Intuition behind convoltion is linear, and time invariant properties of system. Since multipling and shifting does not change the system output or system, we can apply this infinitly many times.
	\par
	
	5)\par
	a)This system is not casual due to output depending on a value at $t+2$
	This system is not stable since $d(x)/dt$ is not bounded if $|x(t)|<B_x$ 
	\par 
	b)This system is casual since output does not depend any future value at $t+n$
	This system is stable because it can be bounded, if $x$ is also bounded.
	\par 
	c)This system is casual since output does not depend any future value at $t+n$
	$$\int_{-\infty}^{\inf}e^{-(t-5)}u(t-5)=\int_{t=5}^{\infty}e^{-(t-5)}=-e^{5-t}=-e^5/e^t$$
	$$lim_{t\rightarrow \infty}-e^5/e^t=0$$
	$$-e^5/e^t<0$$
	This system can be bounded therefore it is stable
	\par 
	d)This system is casual since output does not depend any future value at $t+n$
	$$\int_{t=-\infty}^{\infty}u(t)-e^{-3t}u(t)=\int_{t=0}^{\infty}1-e^{-3t}=1+e^{\infty}-2=\infty$$
	This system can not be bounded therefore it is not stable
	\par
	
	6)Code for this question can be found in attachment.
	\par 
	Result: y[n]=[ 6 10 18 16 18 12  8  2]

	\par
	
	7)$$y[n]=\sum_{k=0}^{1}h[k]x[n-k]$$
	$$y[0]=2.3=6$$
	$$y[1]=4.3+2.(-1)=10$$
	$$y[2]=6.3+4.(-1)+2.2=18$$
	$$y[3]=4.3+6.(-1)+4.2+2.1=16$$
	$$y[4]=2.3+4.(-1)+6.2+4.1=18$$
	$$y[5]=2.(-1)+4.2+6.1=12$$
	$$y[6]=2.2+4.1=8 $$
	$$y[7]=2.1=2 $$
	\par
	
	8)convolve2d function of Scipy, performs convolution operation between 2 dimensional matrices. Applying convolution on each channel of the image (R,G,B) with smoothing box filter creates a blur effect on the image.
	Code for this question can be found in attachment.	
	\par
\end{document}